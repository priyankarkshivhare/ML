\section{Seaborn Visualization Library}

Seaborn is a Python data visualization library based on Matplotlib. It provides a high-level interface for drawing attractive and informative statistical graphics.

\subsection{Scatter Plot}

\textbf{Definition:} A scatter plot displays the relationship between two numerical variables.

\textbf{Function:} \texttt{seaborn.scatterplot()}

\textbf{Syntax:}
\begin{verbatim}
sns.scatterplot(data=df, x='x_column', y='y_column', hue='category',
                style='marker_style', size='size_variable')
\end{verbatim}

\textbf{Key Parameters:}
\begin{itemize}
    \item \texttt{x}, \texttt{y}: Variables to plot on the x and y axes.
    \item \texttt{hue}: Categorical variable that maps colors.
    \item \texttt{style}: Categorical variable for marker style.
    \item \texttt{size}: Numerical or categorical variable for marker size.
\end{itemize}

\textbf{Example:}
\begin{verbatim}
sns.scatterplot(data=tips, x='total_bill', y='tip', hue='sex', style='smoker')
\end{verbatim}

\textbf{Best Practices:}
\begin{itemize}
    \item Use \texttt{hue}, \texttt{style}, and \texttt{size} to represent additional dimensions.
    \item Use \texttt{alpha} to reduce overplotting.
\end{itemize}

\subsection{Distribution Plots}

\subsubsection{Rug Plot}
\textbf{Definition:} Displays individual data points as small vertical ticks.

\textbf{Function:} \texttt{seaborn.rugplot()}

\textbf{Syntax:}
\begin{verbatim}
sns.rugplot(data=df, x='variable')
\end{verbatim}

\textbf{Use Case:} Often used alongside KDE or histogram plots for highlighting exact data points.

\subsubsection{Hist Plot}
\textbf{Definition:} Plots the frequency distribution of a numeric variable.

\textbf{Function:} \texttt{seaborn.histplot()}

\textbf{Syntax:}
\begin{verbatim}
sns.histplot(data=df, x='variable', bins=30, kde=True)
\end{verbatim}

\textbf{Key Parameters:}
\begin{itemize}
    \item \texttt{bins}: Number of histogram bins.
    \item \texttt{kde}: Add a KDE curve if \texttt{True}.
\end{itemize}

\subsubsection{Dis Plot}
\textbf{Definition:} A figure-level interface to draw distribution plots with optional facets.

\textbf{Function:} \texttt{seaborn.displot()}

\textbf{Syntax:}
\begin{verbatim}
sns.displot(data=df, x='variable', kind='hist', col='category')
\end{verbatim}

\textbf{Kind Options:} \texttt{'hist'}, \texttt{'kde'}, \texttt{'ecdf'}

\textbf{Best Practices:}
\begin{itemize}
    \item Use \texttt{displot} for comparing distributions across subsets.
    \item Combine histogram with KDE for a complete picture.
\end{itemize}

\subsection{Categorical Plots}

\subsubsection{Box Plot}
\textbf{Definition:} Shows median, quartiles, and outliers of a distribution.

\textbf{Function:} \texttt{seaborn.boxplot()}

\textbf{Syntax:}
\begin{verbatim}
sns.boxplot(data=df, x='category', y='value')
\end{verbatim}

\textbf{Use Case:} Best for summarizing numeric data grouped by category.

\subsubsection{Violin Plot}
\textbf{Definition:} Combines box plot and KDE for detailed distribution.

\textbf{Function:} \texttt{seaborn.violinplot()}

\textbf{Syntax:}
\begin{verbatim}
sns.violinplot(data=df, x='category', y='value')
\end{verbatim}

\textbf{Use Case:} Use when the shape of the distribution is important.

\subsubsection{Swarm Plot}
\textbf{Definition:} Displays all data points while avoiding overlap.

\textbf{Function:} \texttt{seaborn.swarmplot()}

\textbf{Syntax:}
\begin{verbatim}
sns.swarmplot(data=df, x='category', y='value')
\end{verbatim}

\subsubsection{Boxen Plot}
\textbf{Definition:} Also known as letter-value plot, suited for large datasets.

\textbf{Function:} \texttt{seaborn.boxenplot()}

\textbf{Syntax:}
\begin{verbatim}
sns.boxenplot(data=df, x='category', y='value')
\end{verbatim}

\textbf{Use Case:} Use to visualize heavy-tailed distributions.

\subsection{Comparison Plots}

\subsubsection{Joint Plot}
\textbf{Definition:} Combines scatterplot with univariate histograms.

\textbf{Function:} \texttt{seaborn.jointplot()}

\textbf{Syntax:}
\begin{verbatim}
sns.jointplot(data=df, x='x', y='y', kind='scatter')
\end{verbatim}

\textbf{Kind Options:} \texttt{'scatter'}, \texttt{'kde'}, \texttt{'hex'}

\subsubsection{Pair Plot}
\textbf{Definition:} Pairwise relationships between multiple variables.

\textbf{Function:} \texttt{seaborn.pairplot()}

\textbf{Syntax:}
\begin{verbatim}
sns.pairplot(data=df, hue='category')
\end{verbatim}

\textbf{Use Case:} Great for EDA on numeric datasets.

\subsection{Grid Plots}

\subsubsection{Cat Plot}
\textbf{Definition:} General categorical plot with subplot support.

\textbf{Function:} \texttt{seaborn.catplot()}

\textbf{Syntax:}
\begin{verbatim}
sns.catplot(data=df, x='category', y='value', kind='box', col='group')
\end{verbatim}

\textbf{Kind Options:} \texttt{'strip'}, \texttt{'swarm'}, \texttt{'box'}, \texttt{'violin'}, \texttt{'bar'}, etc.

\subsubsection{PairGrid}
\textbf{Definition:} Fully customizable pairwise plot grid.

\textbf{Function:} \texttt{seaborn.PairGrid()}

\textbf{Syntax:}
\begin{verbatim}
g = sns.PairGrid(df)
g.map_upper(sns.scatterplot)
g.map_lower(sns.kdeplot)
g.map_diag(sns.histplot)
\end{verbatim}

\textbf{Use Case:} Use when you need control over each subplot.

\subsubsection{FacetGrid}
\textbf{Definition:} Allows plotting conditional relationships.

\textbf{Function:} \texttt{seaborn.FacetGrid()}

\textbf{Syntax:}
\begin{verbatim}
g = sns.FacetGrid(df, col='col_variable', row='row_variable', hue='hue_variable')
g.map(sns.scatterplot, 'x', 'y')
\end{verbatim}

\subsection{Matrix Maps}

\subsubsection{Heatmap}
\textbf{Definition:} Encodes matrix values with colors.

\textbf{Function:} \texttt{seaborn.heatmap()}

\textbf{Syntax:}
\begin{verbatim}
sns.heatmap(data=matrix_data, annot=True, fmt='d', cmap='YlGnBu')
\end{verbatim}

\textbf{Key Parameters:}
\begin{itemize}
    \item \texttt{annot}: Show cell values.
    \item \texttt{fmt}: Value formatting string.
    \item \texttt{cmap}: Color palette (e.g., \texttt{'coolwarm'}, \texttt{'YlGnBu'}).
\end{itemize}

\subsubsection{Clustermap}
\textbf{Definition:} Heatmap + hierarchical clustering.

\textbf{Function:} \texttt{seaborn.clustermap()}

\textbf{Syntax:}
\begin{verbatim}
sns.clustermap(data=matrix_data, cmap='coolwarm', standard_scale=1)
\end{verbatim}

\textbf{Use Case:} Ideal for finding hidden groupings in complex datasets.

